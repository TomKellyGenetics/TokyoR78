\documentclass{beamer}

\usepackage[utf8]{inputenc}
\usepackage{default}

\hypersetup{pdfstartview={Fit}}
\usepackage{siunitx} % use of \times 10 - X format
\usepackage{colortbl, xcolor}
\usepackage[dvips]{color}
\usepackage[first=0,last=9]{lcg}
\usepackage{hyperref}
\hypersetup{
    bookmarks=true,         % show bookmarks bar?
    unicode=false,          % non-Latin characters in Acrobat’s bookmarks
    pdftoolbar=true,        % show Acrobat’s toolbar?
    pdfmenubar=true,        % show Acrobat’s menu?
    pdffitwindow=false,     % window fit to page when opened
    pdfstartview={FitH},    % fits the width of the page to the window
    pdftitle={My title},    % title
    pdfauthor={Author},     % author
    pdfsubject={Subject},   % subject of the document
    pdfcreator={Creator},   % creator of the document
    pdfproducer={Producer}, % producer of the document
    pdfkeywords={keyword1, key2, key3}, % list of keywords
    pdfnewwindow=true,      % links in new PDF window
    colorlinks=true,       % false: boxed links; true: colored links
    linkcolor=red,          % color of internal links (change box color with linkbordercolor)
    citecolor=green,        % color of links to bibliography
    filecolor=magenta,      % color of file links
    urlcolor=blue           % color of external links
}
%\usepackage{animate}
\usepackage{adjustbox}

\definecolor{Black}{rgb}{1, 1, 1}

\definecolor{Set1}{rgb}{1, 0.9411765, 0.9411765}
\definecolor{Set2}{rgb}{0.9411765,0.9411765, 1}
\definecolor{Intersect}{rgb}{0.9411765, 0.8862745, 0.9411765}

\definecolor{Cluster_Blue}{rgb}{0, 0, 1}
\definecolor{Cluster_Green}{rgb}{0, 1, 0}
\definecolor{Cluster_Orange}{rgb}{1, 0.6470588, 0}
\definecolor{Cluster_Red}{rgb}{1, 0, 0}

\definecolor{LightGrey}{rgb}{0.9, 0.9, 0.9}

%enable fixed column size tables
\usepackage{array}
\newcolumntype{H}{>{\setbox0=\hbox\bgroup\cellcolor{white}}c<{\egroup}@{}}
\usepackage{ragged2e}
\newcolumntype{L}[1]{>{\raggedright\let\newline\\\arraybackslash\hspace{0pt}}m{#1}}
\newcolumntype{C}[1]{>{\centering\let\newline\\\arraybackslash\hspace{0pt}}m{#1}}
\newcolumntype{R}[1]{>{\raggedleft\let\newline\\\arraybackslash\hspace{0pt}}m{#1}}

\usepackage{multirow}

\usepackage[iso]{datetime2}

\usepackage{CJKutf8}

\setbeamertemplate{frametitle}[default][center]
\setbeamertemplate{caption}{\raggedright\insertcaption\par}
%\setbeamertemplate{framesubtitle}[default][center]

\begin{document}
\begin{frame}
    \frametitle{Making your code faster:   \\
    \begin{CJK}{UTF8}{min}コードを早く実行させている\end{CJK}}
    \framesubtitle{Introduction to vectorisation and parallel computing \\
    \begin{CJK}{UTF8}{min}ベクトル化と並列計算の紹介する\end{CJK}}
    \begin{center}
    \texttt{\#}TokyoR 78\textsuperscript{\small{th}} Meeting \\
    \changeformat{2019-05-25} \\
      %\changeformat{\today} \\
      \bigskip
      \resizebox{!}{0.25 \textheight}{
	\includegraphics{RIKEN_logo.png}
      }   
    \end{center}
    \newline
    \begin{center}
        \textbf{Tom Kelly} \\
        \small{
      Postdoctoral Researcher \\
      %Epigenome Technology Exploration Unit, %\\
      %Division of Genomic Medicine \\
      RIKEN Centre for Integrative Medical Sciences, %\\
      Yokohama%, Japan
      }
    \end{center}        
    \\
     \begin{center}
     \begin{CJK}{UTF8}{min}
      \textbf{ケリー・トム} \\
      \small{
      ポスドクで 研究者 \\
      %エピゲノム技術開発ユニット、 %\\ 
      %Division of Genomic Medicine \\
      国立研究開発法人理化学研究所の生命医科学研究センター、 %\\
      横浜%、日本
      }
      \end{CJK}
    \end{center}    
    
    \end{frame}
  \begin{frame}
    \frametitle{ \begin{CJK}{UTF8}{min}自己紹介\end{CJK}}
    \framesubtitle{ \begin{CJK}{UTF8}{min}ケリー・トム\end{CJK}}
     \begin{columns}[T] % contents are top vertically aligned
  \begin{column}[T]{7.5cm} % each column can also be its own environment  
  
       \begin{CJK}{UTF8}{min}
     \begin{itemize}
     
     \item 名前はトム
      \bigskip
     \item 苗字はケリー
      \bigskip
	\item ニュージーランドから来た
      \bigskip
     \item 二十七歳
      \bigskip
	\item 妻の出身は東北
	\bigskip
	\item 一年半ぐらい日本に住んでいた\\ (宮城県と神奈川県)
     \end{itemize}
      \end{CJK}
     
     \end{column}
     \begin{column}[T]{4cm} 
      \vskip-3cm
      \begin{center}
      \resizebox{0.75 \columnwidth}{!}{
	\includegraphics{Otago_Uni.jpg}
      }
     
     \smallskip     
     
      \resizebox{0.75 \columnwidth}{!}{
	\includegraphics{pohutu-geyser.jpg}
      }
     
      \smallskip     
     
      \resizebox{0.75  \columnwidth}{!}{
	\includegraphics{milford-sound.jpg}
      }
     
      \footnotesize{
      

      Twitter: \texttt{@}tomkXY
      
      \smallskip

       GitHub:TomKellyGenetics  
       }
    \end{center}
    
     \end{column}
     \end{columns}
  
  \end{frame}
  \begin{frame}
    \frametitle{ \begin{CJK}{UTF8}{min}自己紹介\end{CJK}}
    \framesubtitle{ \begin{CJK}{UTF8}{min}ケリー・トム\end{CJK}}
     \begin{columns}[T] % contents are top vertically aligned
  \begin{column}[T]{7.5cm} % each column can also be its own environment  
  
       \begin{CJK}{UTF8}{min}
     \begin{itemize}
	\item 専攻は遺伝学と数学
      \bigskip
     \item バイオインフォマティクスの研究者 
     
     \bigskip
     \item 主に統計解析を行う
     
      \bigskip
     \item オタゴ大学で博士号を取った
      \bigskip
     \item 横浜で理化学研究所でポスドク     
     
     \bigskip
     \item R言語を7年間ぐらい使っている
     \end{itemize}
      \end{CJK}
     
     \end{column}
     \begin{column}[T]{4cm} 
      \vskip-3cm
      \begin{center}
      \resizebox{0.75 \columnwidth}{!}{
	\includegraphics{Otago_Uni.jpg}
      }
     
     \smallskip     
     
      \resizebox{0.75 \columnwidth}{!}{
	\includegraphics{pohutu-geyser.jpg}
      }
     
      \smallskip     
     
      \resizebox{0.75  \columnwidth}{!}{
	\includegraphics{milford-sound.jpg}
      }
     
      \footnotesize{
      

      Twitter: \texttt{@}tomkXY
      
      \smallskip

       GitHub:TomKellyGenetics  
       }
    \end{center}
    
     \end{column}
     \end{columns}
  
  \end{frame}
   \begin{frame}
    \frametitle{ R is too slow}
    \framesubtitle{ \begin{CJK}{UTF8}{min}R言語は遅いすぎる\end{CJK}}
    \resizebox{!}{0.66 \textheight}{
\centered{ \includegraphics{computer_guy.png}}
     }
      \end{frame}
    \begin{frame}
    \frametitle{ R is too slow}
    \framesubtitle{ \begin{CJK}{UTF8}{min}R言語は遅いすぎる\end{CJK}}
  
  How to make your code Faster
  
      
     \begin{itemize}
    \item Develop automated complex tasks with \textit{Loops} and then optimise code (Loops are slow)    
           \bigskip
           
	\item Create your own \textit{functions} to automate tasks (remove human-error)
      \bigskip
      
     \item Use built-in \textit{vectorised} and apply functions to process vectors, matrices, and lists more efficiently
     
     \bigskip
     \item Pass functions to other languages (e.g., use C++ with Rcpp)
     
      \bigskip
     \item Run independent tasks in \textit{parallel}
     
      \bigskip
     \item Use remote servers, clusters, and high-performance computing (HPC)     
     
     \end{itemize}
      \end{frame}
      \begin{frame}
    \frametitle{Key point}
      
Running a task with multiple inputs

\begin{CJK}{UTF8}{min}ダメです\end{CJK}
     
     \begin{itemize}
    \item  Don't copy-paste!  (introduce human errors)
    
    \item Only use loops when needed
    
    \item Avoid "premature optimisation" (code first, speed up later)
    
    \end{itemize}
    
    \begin{CJK}{UTF8}{min}オケです\end{CJK}
    \begin{itemize}
    
    \item Write a function when you need to do something more than once 
     \begin{itemize}
       \item run a function with different inputs
       \item share functions in packages
       \end{itemize}
       
       \item evaluate in all elements at once
     \begin{itemize}
       \item vectorised functions
       \item the "apply" functions (and \texttt{plyr} package) 
       \end{itemize}
       
       \item \textit{parallel} computing with \texttt{snow} package
     \begin{itemize}
       \item set up dependent nodes (SOCKS/openmpi)
       \item  export input objects to cluster
       \item run function in parallel (non-sequential) as multiple "threads"
       \end{itemize}
     
     \end{itemize}
      \end{frame}
      
       \begin{frame}
   
   \begin{center}
   
   \LARGE{Demonstration}\\
   \LARGE{\begin{CJK}{UTF8}{min}デモ\end{CJK}}
   
   \end{center}

  \end{frame}  
  
      \begin{frame}
    \frametitle{ Parallel Computing in R}
    \framesubtitle{Take home messages}
  
  How to make your code Faster
  
      
     \begin{itemize}
    \item Identify "bottleneck" points and optimise only this code
    \bigskip
    
   \item Concepts in parallel computing can be applied to other package or languages languages (e.g., \texttt{dopar}, GNU parallel)
    \bigskip
           
	\item Embarrassingly parallel processes \textit{do not} run in order
      \bigskip
      
     \item Parallel computing is \textit{not always} faster (due to cluster set up "overheads" and communication)
     
     \end{itemize}
     
     Demonstration  codes will be released on GitHub as a Rmarkdown document: \texttt{TomKellyGenetics\/TokyoR78}
      \end{frame}
           \begin{frame}
    \frametitle{ Automating tasks in R}
  
  How to run functions at the same time
  
      
     \begin{itemize}
    \item  Running R in the background
     
       \begin{itemize}
       \item  RStudio jobs pane
      \item  running as a script (nohup)
      \item  pass arguments to script: \texttt{commandargs()} in R, \texttt{"\$@"} in bash
      \end{itemize}

     \item Submit jobs to a server
     
     "The Cloud is just someone else's computer"
     
       \begin{itemize}
       \item  Run on a local server or remote cluster (ssh)
      \item   Move input data to server (rsync or scp)
      \item   Make sure dependancies (packages) are installed
      \item   Queue jobs to the server with a scheduler
      \begin{itemize}
      \item LoadLeveler: llq, llsubmit, llcancel
       \item  Slurm: squeue, srun, scancel
        \item SGE:   qstat, qsub, qdel 
        \end{itemize}
      
      \item Run parallel scripts
      \item Dependancy jobs (run after other jobs have finished)

      \end{itemize}
     \end{itemize}
     
     See StackOverflow answer for more details
      \end{frame}
      
            \begin{frame}
            \begin{center}
            
            \resizebox{!}{0.9 \textheight}{      
	\includegraphics{Stack_Question.png}
      }
            \end{center}
      
      \small \url{https://stackoverflow.com/questions/31137842/run-multiple-r-scripts-simultaneously/41920899#41920899}
              \end{frame}
           
            \begin{frame}
    \frametitle{ Running R on remote systems}
    \framesubtitle{Take home messages}
  
  How to make your code Faster
      
     \begin{itemize}
    \item  Ask about servers/clusters available and take opportunity to learn use them
    \bigskip
    
   \item Be careful of "pre-mature optimisation"
    \bigskip
    
   \item Automate tasks to save you work and perform reproducible analysis
    \bigskip
    
   \item Run small "test" jobs to check it will run without errors
    \bigskip
    
   \item Order of executing tasks is important
    \bigskip
    
   \item CPU-hours cost money (but not as much as human-hours)
   
     
     \end{itemize}
     
     Demonstration  codes will be released on GitHub as a Rmarkdown document
      \end{frame}
  \iffalse
  \begin{frame}
  \frametitle{Do you know about New Zealand? \\
  \begin{CJK}{UTF8}{min}ニュージーランドについて知っている?\end{CJK}}  
  
  \begin{columns}[T] % contents are top vertically aligned
  \begin{column}[T]{4cm} %

   \begin{center}
      \resizebox{0.75 \columnwidth}{!}{      
	\includegraphics{Flag_of_New_Zealand.png}
      }
      
      \bigskip
      
        
      \resizebox{0.6 \columnwidth}{!}{      
	\includegraphics{Silver_Fern_Real.png}
      }
      
      \bigskip
      
      
      \resizebox{0.75 \columnwidth}{!}{      
	\includegraphics{Silver_Fern_Flag.png}
      }
       \end{center}
     \end{column}
         \begin{column}[T]{4cm} %
     \begin{center}

      \resizebox{0.75 \columnwidth}{!}{
	\includegraphics{Maori_Flag.png}
      }
      
      \bigskip
      
        
      \resizebox{0.6 \columnwidth}{!}{      
	\includegraphics{Koru_Real.png}
      }
      
      \bigskip
      
      
      \resizebox{0.75 \columnwidth}{!}{      
	\includegraphics{AirNZ_Plane.png}
      }
       \end{center}
      \end{column}
         \begin{column}[T]{4cm} %    
      \begin{center}
      \resizebox{0.75  \columnwidth}{!}{
	\includegraphics{Red_Peak.png}
      }
      
      \bigskip
      
        
      \resizebox{0.6 \columnwidth}{!}{      
	\includegraphics{Mitre_Peak.png}
      }
      
      \bigskip
      
        
      \resizebox{0.75 \columnwidth}{!}{      
	\includegraphics{All_Blacks.png}
      }
    \end{center}
       \end{column}
     \end{columns}  
  
  \end{frame}
  \begin{frame}
  \frametitle{In New Zealand we have many cultures \\
  \begin{CJK}{UTF8}{min}ニュージーランドでいろいろい文化がある\end{CJK}}  
  
  \begin{columns}[T] % contents are top vertically aligned
  \begin{column}[T]{4cm} %

   \begin{center}
   New Zealand European \\
\begin{CJK}{UTF8}{min}ニュージーランドの \\ ユーロパ人\end{CJK}   \\
P\={a}keh\={a} \\
(\begin{CJK}{UTF8}{min}外国人\end{CJK})

\bigskip

      \resizebox{0.75 \columnwidth}{!}{      
	\includegraphics{Profile.png}
      }
       \end{center}
     \end{column}
         \begin{column}[T]{4cm} %
     \begin{center}
        New Zealand M\={a}ori \\
\begin{CJK}{UTF8}{min}ニュージーランドの \\ マオリ人\end{CJK}    \\
M\={a}ori; T\={a}ngata Whenua (\begin{CJK}{UTF8}{min}先住民; 土地の人々\end{CJK}) \\

\bigskip

      \resizebox{0.75 \columnwidth}{!}{
	\includegraphics{Maori.pdf}
      }
       \end{center}
      \end{column}
         \begin{column}[T]{4cm} %    
      \begin{center}
      Hobbits \\
\begin{CJK}{UTF8}{min}ホビット\end{CJK}   \\
From Middle Earth \\
 (\begin{CJK}{UTF8}{min}中っ国人;シャイア人\end{CJK}) \\
 \begin{CJK}{UTF8}{min}ロードオブザリング\end{CJK}

 
\bigskip

 
      \resizebox{0.75  \columnwidth}{!}{
	\includegraphics{Frodo_Ending.png}
      }
    \end{center}
       \end{column}
     \end{columns}  
  
  \end{frame}
  \begin{frame}
  \frametitle{We are all "Kiwi"\\
  \begin{CJK}{UTF8}{min}全部ニュージーランド人は「キウィ」\end{CJK}}  
  
  \begin{columns}[T] % contents are top vertically aligned
  \begin{column}[T]{5cm} %

   \begin{center}
   Kiwi 
\begin{CJK}{UTF8}{min}キウィ\end{CJK}   \\

\smallskip

      \resizebox{0.6 \columnwidth}{!}{      
	\includegraphics{Profile.png}
      }
      
      \bigskip
      
      Kiwifruit
\begin{CJK}{UTF8}{min}キウィフルーツ\end{CJK}   \\

\smallskip

      \resizebox{0.6 \columnwidth}{!}{      
	\includegraphics{Kiwifruit.jpg}
      }
       \end{center}
     \end{column}
         \begin{column}[T]{5cm} %
     \begin{center}
        Kiwi 
\begin{CJK}{UTF8}{min}キウィ\end{CJK}   \\

\smallskip

      \resizebox{0.6 \columnwidth}{!}{
	\includegraphics{Maori.pdf}
      }
      
      \bigskip
      
      Kiwi 
\begin{CJK}{UTF8}{min}キウィ鳥\end{CJK}   \\

\smallskip

      \resizebox{0.6 \columnwidth}{!}{      
	\includegraphics{meet-kiwi.jpg}
      }
      \end{center}
      \end{column}
     \end{columns}  
  
  \end{frame}
   \begin{frame}
   
   \begin{center}
   
   \large{T\={e}n\={a} koutou, t\={e}n\={a} koutou, t\={e}n\={a} koutou katoa }\\
   \bigskip
   \large{Hello guests, hello everyone, hello spirits looking down upon us}\\   
   \bigskip
   \large{\begin{CJK}{UTF8}{min}皆さん、初めまして、よろしくお願いします。\end{CJK}}
   
   \end{center}

  \end{frame}
 \begin{frame}
  \frametitle{M\={a}ori Culture and Language \\
  \begin{CJK}{UTF8}{min}マオリ文化とマオリ語\end{CJK}}  
  
  \begin{columns}[T] % contents are top vertically aligned
  \begin{column}[T]{9cm} %
  \small{

      T\={e}n\={a} koutou, t\={e}n\={a} koutou, t\={e}n\={a} koutou katoa  \\ Greetings friends, family, guests, and ancestors/spirits   \\ \begin{CJK}{UTF8}{min}皆さん、初めまして、よろしくお願いします。\end{CJK}
      \bigskip
       
    \begin{columns}[T] % contents are top vertically aligned
  \begin{column}[T]{4cm} %     
  Kia Ora (Hello) \\ \begin{CJK}{UTF8}{min}こんにちは\end{CJK}
      \bigskip
      
Kia Kaha (Be strong) \\ \begin{CJK}{UTF8}{min}頑張って!\end{CJK}
      \bigskip
   
Koha (Gift) \\ \begin{CJK}{UTF8}{min}おみやげ\end{CJK}
      \bigskip
   
Mana (Reputation) \\ \begin{CJK}{UTF8}{min}心や気\end{CJK}
      \bigskip
   
Taku h\={e} (Sorry) \\ \begin{CJK}{UTF8}{min}すみません\end{CJK}
      
       \end{column}
        \begin{column}[T]{4cm}
   
Haere mai! (Welcome) \\ \begin{CJK}{UTF8}{min}ようこそ\end{CJK}
      \bigskip
   
Whanau (Family) \\ \begin{CJK}{UTF8}{min}家\end{CJK}
      \bigskip
   
Aroha (Love) \\ \begin{CJK}{UTF8}{min}愛\end{CJK}
      \bigskip
   
Ka kite ano (See you) \\ \begin{CJK}{UTF8}{min}またね!\end{CJK}
      \bigskip
   
  Nga mihi (Thank you) \\ \begin{CJK}{UTF8}{min}どうもありがとう\end{CJK}

   \end{column}
   \end{columns}
}
       \end{column} 
       \begin{column}[T]{3cm} %
      
       %\vskip2m
\begin{center}
      \resizebox{1 \columnwidth}{!}{
	\includegraphics{Maori.pdf}
      }   
      
      \bigskip
      
      \resizebox{1 \columnwidth}{!}{
	\includegraphics{Maui.jpg}
      }   
  \end{center}
       \end{column}
     \end{columns}  
  
  \end{frame}

   \begin{frame}
   
   \begin{center}
   
   \Large{He aha te mea nui o te ao? }\\
   \Large{He t\={a}ngata,  he t\={a}ngata, he t\={a}ngata.}\\
   
   \textcolor{white}{
  \bigskip   
   \Large{What is the most important thing in the world?}\\   
   \Large{It is people, it is people, it is people.}\\
   
\bigskip   
   
   \Large{\begin{CJK}{UTF8}{min}\textcolor{white}{世界で一番重要なことは何ですか?}\end{CJK}}\\
   \Large{\begin{CJK}{UTF8}{min}\textcolor{white}{人々だし、人々だし、人々です。}\end{CJK}}
   }
   \end{center}

  \end{frame}  
  
   \begin{frame}
   
   \begin{center}
   
   \Large{He aha te mea nui o te ao? }\\
   \Large{He t\={a}ngata,  he t\={a}ngata, he t\={a}ngata.}\\
   
   
  \bigskip   
   \Large{What is the most important thing in the world?}\\   
   \textcolor{white}{
   \Large{It is people, it is people, it is people.}\\
   }
   
\bigskip   
   
   \Large{\begin{CJK}{UTF8}{min}世界で一番重要なことは何ですか?\end{CJK}}\\
   \Large{\begin{CJK}{UTF8}{min}\textcolor{white}{人々だし、人々だし、人々です。}\end{CJK}}
   
   \end{center}

  \end{frame}  
     \begin{frame}
   
   \begin{center}
   
   \Large{He aha te mea nui o te ao? }\\
   \Large{He t\={a}ngata,  he t\={a}ngata, he t\={a}ngata.}\\
   
   
  \bigskip   
   \Large{What is the most important thing in the world?}\\   
   \Large{It is people, it is people, it is people.}\\
   
\bigskip   
   
   \Large{\begin{CJK}{UTF8}{min}世界で一番重要なことは何ですか?\end{CJK}}\\
   \Large{\begin{CJK}{UTF8}{min}人々だし、人々だし、人々です。\end{CJK}}
   
   \end{center}

  \end{frame}  
  
  \begin{frame}
  \frametitle{Why would you learn R? \\
  \begin{CJK}{UTF8}{min}どうしてR言語を学びたい?\end{CJK}}  
  
  \begin{columns}[T] % contents are top vertically aligned
  \begin{column}[T]{6cm} %

       \begin{itemize}
  \item Free and open-source \\ \begin{CJK}{UTF8}{min}無料とオープンソース\end{CJK}
      \smallskip
   
   \item Global user community \\ \begin{CJK}{UTF8}{min}国際ユーザコミュニティ\end{CJK}
      \smallskip
   
   \item A huge ecosystem of packages \\ \begin{CJK}{UTF8}{min}多くのパッケージがある\end{CJK}
      \smallskip
   
   \item Reproducible research \\ \begin{CJK}{UTF8}{min}再現可能な研究\end{CJK}
   \end{itemize}
       \end{column} 
       \begin{column}[T]{6cm} %
      
       %\vskip2m
\begin{center}
      \resizebox{0.66 \columnwidth}{!}{
	\includegraphics{R_Logo.jpg}
      }   
  \end{center}
       \end{column}
     \end{columns}  
  
  \end{frame}
  \begin{frame}
   
   \begin{center}
   
   \LARGE{What is the best way to learn?}\\
   \LARGE{\begin{CJK}{UTF8}{min}学ぶのに一番良い方法は何?\end{CJK}}
   
   \end{center}

  \end{frame}
  \begin{frame}
   
   \begin{center}
   
   \LARGE{Teaching}\\
   \LARGE{\begin{CJK}{UTF8}{min}教える事\end{CJK}}
   
   \end{center}

  \end{frame}
  \begin{frame}
   
   \begin{center}
   
   \LARGE{Practice on your own}\\
   \LARGE{\begin{CJK}{UTF8}{min}自分で練習する\end{CJK}}
   
   \end{center}

  \end{frame}  
    \begin{frame}
   
   \begin{center}
   
   \LARGE{Get colleagues and the community to help}\\
   \LARGE{\begin{CJK}{UTF8}{min}同僚やコミュニティーを支援\end{CJK}}
   
   \end{center}

  \end{frame}
    \begin{frame}

   \frametitle{What is the best way to learn? \\
   \begin{CJK}{UTF8}{min}学ぶのに一番良い方法は何?\end{CJK}}
   
   \begin{center}
   
   
\bigskip
   
   \LARGE{Teaching}\\
   \LARGE{\begin{CJK}{UTF8}{min}教える事\end{CJK}}
   
\bigskip
   
   \LARGE{Practice on your own}\\
   \LARGE{\begin{CJK}{UTF8}{min}自分で練習する\end{CJK}}

\bigskip
   
   \LARGE{Get colleagues and the community to help}\\
   \LARGE{\begin{CJK}{UTF8}{min}同僚やコミュニティーを支援\end{CJK}}
   
   \end{center}

  \end{frame}
  \begin{frame}
   
   \begin{center}
   
   \LARGE{What is ``Software Carpentry''?}\\
   \LARGE{\begin{CJK}{UTF8}{min}「ソフトウェア・カーペントリー」は何ですか?\end{CJK}}
   
   
   \begin{center}
      \resizebox{0.5 \columnwidth}{!}{
	\includegraphics{software-carpentry.png}
      }   
  \end{center}
   \end{center}

  \end{frame}
  
   \begin{frame}
   
   \begin{center}
   
   \LARGE{What is ``Software Carpentry''?}\\
   \LARGE{\begin{CJK}{UTF8}{min}「ソフトウェア・カーペントリー」は何ですか?\end{CJK}}
   
   
   \begin{center}
      \resizebox{0.75 \columnwidth}{!}{
	\includegraphics{ToolKit.png}
      }   
  \end{center}
   \end{center}

  \end{frame}
  \fi
    \begin{frame}
  \frametitle{What is ``Software Carpentry''? \\
  \begin{CJK}{UTF8}{min}「ソフトウェア・カーペントリー」は何ですか?\end{CJK}}  
  
  \vskip-2cm
\begin{columns}[T] % contents are top vertically aligned
  \begin{column}[T]{7.5cm} %
       \end{column} \begin{column}[T]{4cm} %
      
\begin{center}
      \resizebox{0.5 \columnwidth}{!}{
	\includegraphics{software-carpentry.png}
      }   
  \end{center}
       \end{column}
     \end{columns}  
  
  \begin{itemize}
  \item A non-profit foundation based in the USA \\ \begin{CJK}{UTF8}{min}アメリカにある非営利団体\end{CJK}
      \smallskip
   
   \item A collection of collaboratively maintained lessons\\ \begin{CJK}{UTF8}{min}協働でレッスンを維持\end{CJK}
      \smallskip
   
   \item A 2-3 day series of hands-on workshops on tools for researchers \\ \begin{CJK}{UTF8}{min}研究者向けツールのハンズオン(体験型)ワークショップを2・3日間で教える\end{CJK}
      \smallskip
   
   \item A global community of instructors and member organisations \\ \begin{CJK}{UTF8}{min}指導者とメンバーからなる組織の国際的なコミュニティー\end{CJK}
   
  
  \end{itemize}
  
  \end{frame}
  \iffalse
\begin{frame}

\begin{center}
      \resizebox{0.9 \columnwidth}{!}{
	\includegraphics{ResBaz2015Team.jpg}
      }   
  \end{center}

\end{frame}
 \begin{frame}
  \frametitle{``Software Carpentry'' workshops \\
  \begin{CJK}{UTF8}{min}「ソフトウェア・カーペントリー」のワークショップ\end{CJK}}  
  
  \vskip-1.5cm
\begin{columns}[T] % contents are top vertically aligned
  \begin{column}[T]{7.5cm} %
       \end{column} \begin{column}[T]{4cm} %
      
\begin{center}
      \resizebox{0.5 \columnwidth}{!}{
	\includegraphics{software-carpentry.png}
      }   
  \end{center}
       \end{column}
     \end{columns}  
  
  \begin{itemize}
  \item Open to anyone \\ \begin{CJK}{UTF8}{min}誰でも参加できる\end{CJK}
      \smallskip
   
   \item Free or small charge for catering/venue \\ \begin{CJK}{UTF8}{min}無料か少額\end{CJK}
      \smallskip
   
   \item Survey for before and after lessons \\ \begin{CJK}{UTF8}{min}レッソの前後にでインケートがある\end{CJK}
      \smallskip
   
   \item Feedback is important to improve lessons\\ \begin{CJK}{UTF8}{min}改善のためのフィードバックは重要です\end{CJK}
        \smallskip
   
   \item Held all over the world \\ \begin{CJK}{UTF8}{min}世界中でたくさんイベントがある\end{CJK}
  
  \end{itemize}
  
  \end{frame}
 \begin{frame}
 
 \begin{center}
      \resizebox{0.9 \columnwidth}{!}{
	\includegraphics{SWC_Instructors_Map}
      }   
      
       Software Carpentry Instructors \\ \begin{CJK}{UTF8}{min}ソフトウェア・カーペントリーのインストラクター\end{CJK}
  \end{center}
 
\end{frame}  

  \begin{frame}

\begin{center}
      \resizebox{0.9 \columnwidth}{!}{
	\includegraphics{ResBazDun2.jpg}
      }   
  \end{center}

\end{frame}
\fi
 \begin{frame}
  \frametitle{What is``Software Carpentry''\\
  \begin{CJK}{UTF8}{min}ソフトウェア・カーペントリー」は何ですか\end{CJK}}  
  
  \vskip-1.5cm
\begin{columns}[T] % contents are top vertically aligned
  \begin{column}[T]{7.5cm} %
       \end{column} \begin{column}[T]{4cm} %
      
\begin{center}
      \resizebox{0.5 \columnwidth}{!}{
	\includegraphics{software-carpentry.png}
      }   
  \end{center}
       \end{column}
     \end{columns}  
  
  Organisations want to build research capacity \\ \begin{CJK}{UTF8}{min}組織は研究能力を構築したい\end{CJK}  
  
  \begin{itemize}
  \item Consulting doesn't scale \\ \begin{CJK}{UTF8}{min}コンサルティングはスケールができない\end{CJK}
      \smallskip
   
   \item Tech support can't help with all research tools \\ \begin{CJK}{UTF8}{min}技術サポートは全部の研究ツールを手伝えない\end{CJK}
      \smallskip
   
   \item You get frustrated and isolated in online courses (MOOCs) \\ \begin{CJK}{UTF8}{min}オンラインコースでイライラして孤独になる\end{CJK}

  \end{itemize}
  
  We aim to build a community and peer support \\ \begin{CJK}{UTF8}{min}コミュニティとピアサポートを構築することを目指している\end{CJK}
  
  \end{frame}
  \iffalse
\begin{frame}

\begin{center}
      \resizebox{0.9 \columnwidth}{!}{
	\includegraphics{ResBaz_Lesson.jpg}
      }   
  \end{center}

\end{frame}
  \begin{frame}
  \frametitle{``Software Carpentry'' lessons \\
  \begin{CJK}{UTF8}{min}「ソフトウェア・カーペントリー」のレッスン\end{CJK}}  
  
  \vskip-0.5cm
\begin{columns}[T] % contents are top vertically aligned
  \begin{column}[T]{7.5cm} %
       \end{column} \begin{column}[T]{4cm} %
      
\begin{center}
      \resizebox{0.5 \columnwidth}{!}{
	\includegraphics{software-carpentry.png}
      }   
  \end{center}
       \end{column}
     \end{columns}  
      \vskip-1cm
  
  \begin{itemize}
  \item Fun and social \\ \begin{CJK}{UTF8}{min}楽しく、気さく\end{CJK}
      \smallskip
   
   \item Busy and intense \\ \begin{CJK}{UTF8}{min}忙しく、賑やかで、活気がある\end{CJK}
      \smallskip
   
   \item An introduction to programming and and best practices \\ \begin{CJK}{UTF8}{min}プログラミングとベスト・プラクティスへの入門\end{CJK}
      \smallskip
   
   \item An introduction help systems and open source community \\ \begin{CJK}{UTF8}{min}ヘルプ・システムやオープン・ソースの入門\end{CJK}
      \smallskip
   
   \item A networking opportunity \\ \begin{CJK}{UTF8}{min}ネットワーキングの機会\end{CJK}
        \smallskip
   
   \item Open, supportive, and safe for all \\ \begin{CJK}{UTF8}{min}皆さんにとって、オープンで、サポート的で、安全\end{CJK}
   
  
  \end{itemize}
  
  \end{frame}
\begin{frame}

\begin{center}
      \resizebox{0.9 \columnwidth}{!}{
	\includegraphics{ResBazSticky.jpg}
      }   
  \end{center}

\end{frame}
   \begin{frame}
  \frametitle{``Software Carpentry'' lessons \\
  \begin{CJK}{UTF8}{min}「ソフトウェア・カーペントリー」のレッスン\end{CJK}}  
  
    \vskip-1cm
\begin{columns}[T] % contents are top vertically aligned
  \begin{column}[T]{7.5cm} %
       \end{column} \begin{column}[T]{4cm} %
      
\begin{center}
      \resizebox{0.5 \columnwidth}{!}{
	\includegraphics{software-carpentry.png}
      }   
  \end{center}
       \end{column}
     \end{columns}  
      %\vskip-1cm
  
  \begin{itemize}
  \item Instructor leads each session\\ \begin{CJK}{UTF8}{min}インストラクターがそれぞれの入門をレーダーする\end{CJK}
      \smallskip
      
  \item Introduce each topic \\ \begin{CJK}{UTF8}{min}それぞれのトピックを紹介する\end{CJK}
      \smallskip
   
   \item Demonstrations and live-coding \\ \begin{CJK}{UTF8}{min}デモとライブ・コーディング\end{CJK}
      \smallskip
   
   \item Interactive with quiz, exercises and group work  \\ \begin{CJK}{UTF8}{min}そこでクイズやエクササイズやグループ・ワークがあるから、とてもインタラクティブです\end{CJK}
      
  \end{itemize}
  
  \end{frame}

    \begin{frame}

\begin{center}
      \resizebox{0.9 \columnwidth}{!}{
	\includegraphics{ResBazSticky3.jpg}
      }   
  \end{center}

\end{frame}
  
   \begin{frame}
  \frametitle{``Software Carpentry'' lessons \\
  \begin{CJK}{UTF8}{min}「ソフトウェア・カーペントリー」のレッスン\end{CJK}}  
  
    \vskip-2cm
\begin{columns}[T] % contents are top vertically aligned
  \begin{column}[T]{7.5cm} %
       \end{column} \begin{column}[T]{4cm} %
      
\begin{center}
      \resizebox{0.5 \columnwidth}{!}{
	\includegraphics{software-carpentry.png}
      }   
  \end{center}
       \end{column}
     \end{columns}  
      %\vskip-1cm
  
  \begin{itemize}
   \item Helpers to give one-on-one assistance, answer questions, and fix technical problems \\ \begin{CJK}{UTF8}{min}ヘルパーは1:1で手伝う\end{CJK}
        \smallskip
   
   \item Feedback encouraged for organisers and instructors to improve \begin{CJK}{UTF8}{min}オーガナイザーやインストラクターの方々にフィードバックを促す\end{CJK}
           \smallskip
   
   \item Lessons notes to review or continue after the event \\ \begin{CJK}{UTF8}{min}レッスンノートはイベント後にネットで利用可能です\end{CJK}
  
  \end{itemize}
  
  \end{frame}
    \begin{frame}

\begin{center}
      \resizebox{0.9 \columnwidth}{!}{
	\includegraphics{ResBazSticky2.jpg}
      }   
  \end{center}

\end{frame}
    \begin{frame}
  \frametitle{``Software Carpentry'' Team \\
  \begin{CJK}{UTF8}{min}「ソフトウェア・カーペントリー」のチーム\end{CJK}}  
  
  
  Instructor (\begin{CJK}{UTF8}{min}インストラクター\end{CJK})
  \begin{itemize}
  \item Instructor leads each session to introduce each topic \\ \begin{CJK}{UTF8}{min}インストラクターは、各トピックの説明を各セッションで行う\end{CJK}
     % \smallskip
   
   %\item Demonstrations and live-coding \\ \begin{CJK}{UTF8}{min}デモとライブ・コーディング\end{CJK}
   \end{itemize}
      \bigskip
      
    Helper  (\begin{CJK}{UTF8}{min}ヘルパー\end{CJK})
  \begin{itemize}
   \item Assist with exercises  \\ \begin{CJK}{UTF8}{min}エクササイズを手伝う\end{CJK}
      %\smallskip
   
   \item Fix technical errors and install problems \\ \begin{CJK}{UTF8}{min}インストールとエラーを修理する\end{CJK}
  
  \end{itemize}
  
  \begin{center}
      \resizebox{0.25 \columnwidth}{!}{
	\includegraphics{stitch_proud.png}
      }   
      
      Nobody gets left behind \\ \begin{CJK}{UTF8}{min}誰も後に残っていない\end{CJK}
  \end{center}
  
  \end{frame}
      \begin{frame}

\begin{center}
      \resizebox{0.9 \columnwidth}{!}{
	\includegraphics{ResBaz_Team.jpg}
      }   
  \end{center}

\end{frame}
  \begin{frame}
  \frametitle{Who goes to ``Software Carpentry'' lessons? \\
  \begin{CJK}{UTF8}{min}誰が「ソフトウェア・カーペントリー」に行く?\end{CJK}}  
  
  \vskip-1cm
\begin{columns}[T] % contents are top vertically aligned
  \begin{column}[T]{7.5cm} %
       \end{column} \begin{column}[T]{4cm} %
      
\begin{center}
      \resizebox{0.5 \columnwidth}{!}{
	\includegraphics{software-carpentry.png}
      }   
  \end{center}
       \end{column}
     \end{columns}  
  
  \begin{itemize}
  \item Researchers of any career stage \\ \begin{CJK}{UTF8}{min}様々なキャリアの研究者\end{CJK}
      \smallskip
   
  
  \item Researchers in any research field\\ \begin{CJK}{UTF8}{min}様々な分野研究者\end{CJK}
      \smallskip
   
   \item No programming experience needed \\ \begin{CJK}{UTF8}{min}プログラミング経験は不要\end{CJK}
      \smallskip
   
   \item Women and minorities welcome \\ \begin{CJK}{UTF8}{min}女性とマイノリティは大歓迎\end{CJK}
      \smallskip
   
   \item Students or staff \\ \begin{CJK}{UTF8}{min}大学院生や研究員や会社員\end{CJK}
   
  
  \end{itemize}
  
  \end{frame}
    \begin{frame}

\begin{center}
      \resizebox{0.9 \columnwidth}{!}{
	\includegraphics{ResBaz_Lesson3.jpg}
      }   
  \end{center}

\end{frame}
  \begin{frame}
  \frametitle{What does``Software Carpentry'' teach? \\
  \begin{CJK}{UTF8}{min}「ソフトウェア・カーペントリー」で何を教える事?\end{CJK}}  
  
  \vskip-1cm
\begin{columns}[T] % contents are top vertically aligned
  \begin{column}[T]{7.5cm} %
       \end{column} \begin{column}[T]{4cm} %
      
\begin{center}
      \resizebox{0.5 \columnwidth}{!}{
	\includegraphics{software-carpentry.png}
      }   
  \end{center}
       \end{column}
     \end{columns}  
       \vskip-1cm
       
     A core set of tools to do research well \\ \begin{CJK}{UTF8}{min} よく研究するためのツールの中核\end{CJK}
    
  \begin{itemize}
  \item Automation (Bash) \\ \begin{CJK}{UTF8}{min}オートメーション\end{CJK}
      \smallskip
   
   \item Programming (R or Python) \\ \begin{CJK}{UTF8}{min}プログラミング\end{CJK}
      \smallskip
   
   \item Version Control (Git) \\ \begin{CJK}{UTF8}{min}バージョン管理\end{CJK}
      \smallskip
   
   \item Reproducible research \\ \begin{CJK}{UTF8}{min}再現可能な研究\end{CJK}
   
  \end{itemize}
  
       Each lesson is to understand programming concepts \\ \begin{CJK}{UTF8}{min} プログラミングの概念を分かるようになる\end{CJK}
  
  \end{frame}
      \begin{frame}

\begin{center}
      \resizebox{0.9 \columnwidth}{!}{
	\includegraphics{ResBaz_Lesson2.jpg}
      }   
  \end{center}

\end{frame}

  \begin{frame}
  \frametitle{What are ``best practices'' for scientific computing? \\
  \begin{CJK}{UTF8}{min}科学計算の 「ベストプラクティス」とは何?\end{CJK}}  
  
  \vskip-1cm
\begin{columns}[T] % contents are top vertically aligned
  \begin{column}[T]{7.5cm} %
       \end{column} \begin{column}[T]{4cm} %
      
\begin{center}
      \resizebox{0.5 \columnwidth}{!}{
	\includegraphics{software-carpentry.png}
      }   
  \end{center}
       \end{column}
     \end{columns}  
     
     Things to aspire to  \begin{CJK}{UTF8}{min} 目指すもの\end{CJK}
     
     \begin{columns}[T] % contents are top vertically aligned
  \begin{column}[T]{6cm} %
     
  \begin{itemize}
\item Code for Humans (Comment) \\ \begin{CJK}{UTF8}{min} 人間ためのコードを書く\end{CJK}
\smallskip  
  
  \item Record history (commit often) \\ \begin{CJK}{UTF8}{min}履歴の記録(コミット)\end{CJK}
      \smallskip
   
   \item Avoid repetition \\ \begin{CJK}{UTF8}{min}反復の回避\end{CJK}
      \smallskip
   
   \item Anticipate mistakes (testing) \\ \begin{CJK}{UTF8}{min}間違いを予測(テストする)\end{CJK}
      \smallskip
   
   \item Reproducible research \\ \begin{CJK}{UTF8}{min}再現可能なワークフロー\end{CJK}
   
  \end{itemize}
  \end{column} \begin{column}[T]{6cm} %
    
  \begin{itemize}
  \item Avoid premature optimisation \\ \begin{CJK}{UTF8}{min}早すぎる最適化の回避\end{CJK}
      \smallskip
   
   \item Collaborate (with GitHub) \\ \begin{CJK}{UTF8}{min}コラボレーション\end{CJK}
      \smallskip
   
   \item Share and review code \\ \begin{CJK}{UTF8}{min}コードの共有とレビュー\end{CJK}
      \smallskip
   
   \item Use the help systems \\ \begin{CJK}{UTF8}{min}ヘルプシステムの使用\end{CJK}
      \smallskip
   
   \item Clean (``tidy'') data \\ \begin{CJK}{UTF8}{min}データの整理\end{CJK}
   
  \end{itemize}  
  
   \end{column}
     \end{columns}  
  \end{frame}
  
  \begin{frame}

\begin{center}
      \resizebox{0.9 \columnwidth}{!}{
	\includegraphics{ResBaz_Exercise.jpg}
      }   
  \end{center}

\end{frame}

  \begin{frame}
  
  \frametitle{Who are``Software Carpentry'' instructors? \\
  \begin{CJK}{UTF8}{min}「ソフトウェア・カーペントリー」の先生は誰?\end{CJK}}  
  
  \vskip-1cm
\begin{columns}[T] % contents are top vertically aligned
  \begin{column}[T]{7.5cm} %
       \end{column} \begin{column}[T]{4cm} %
      
\begin{center}
      \resizebox{0.5 \columnwidth}{!}{
	\includegraphics{software-carpentry.png}
      }   
  \end{center}
       \end{column}
     \end{columns}  
  
  \begin{itemize}
  \item Volunteers \\ \begin{CJK}{UTF8}{min}ボランティア\end{CJK}
      \smallskip
   
   \item Trained instructors \\ \begin{CJK}{UTF8}{min}訓練を受けたインストラクター\end{CJK}
      \smallskip
   
   \item People passionate about good research \\ \begin{CJK}{UTF8}{min}良い研究に情熱をもつ人々\end{CJK}
      \smallskip
   
   \item Experts in many different fields \\ \begin{CJK}{UTF8}{min}さまざまな分野の専門家\end{CJK}
      \smallskip
   
   \item From all over the world \\ \begin{CJK}{UTF8}{min}世界中から
\end{CJK}
   
  \end{itemize}
  
  \end{frame}
  
\begin{frame}

\begin{center}
      \resizebox{0.9 \columnwidth}{!}{
	\includegraphics{ResBaz2015Team.jpg}
      }   
  \end{center}

\end{frame}
  
  \begin{frame}
   
   \begin{center}
   
   \LARGE{What is in it for me?}\\
   \LARGE{\begin{CJK}{UTF8}{min}利点は何ですか?\end{CJK}}
   
   
   \begin{center}
      \resizebox{0.5 \columnwidth}{!}{
	\includegraphics{software-carpentry.png}
      }   
  \end{center}
   \end{center}

  \end{frame}
  \begin{frame}
  
  \frametitle{Benefits for ``Software Carpentry'' instructors \\
  \begin{CJK}{UTF8}{min}「ソフトウェア・カーペントリー」の利点\end{CJK}}  
  
  \vskip-1cm
\begin{columns}[T] % contents are top vertically aligned
  \begin{column}[T]{7.5cm} %
       \end{column} \begin{column}[T]{4cm} %
      
\begin{center}
      \resizebox{0.5 \columnwidth}{!}{
	\includegraphics{software-carpentry.png}
      }   
  \end{center}
       \end{column}
     \end{columns}  
  
  \begin{columns}[T] % contents are top vertically aligned
  \begin{column}[T]{6cm} %
  \begin{itemize}
  \item Free training \\ \begin{CJK}{UTF8}{min}無料トレーニング\end{CJK}
      \smallskip
   
   \item Networking \\ \begin{CJK}{UTF8}{min}ネットワーキング\end{CJK}
      \smallskip
   
   \item Travel opportunities \\ \begin{CJK}{UTF8}{min}旅行の機会\end{CJK}
      \smallskip
   
   \item Career development and support \\ \begin{CJK}{UTF8}{min}キャリア開発とサポート\end{CJK}
      \smallskip
   
   \item Develop ``soft skills'' \\ \begin{CJK}{UTF8}{min}「ソフトスキル」を開発する
\end{CJK}
   
  \end{itemize}
  \end{column} \begin{column}[T]{6cm} %
    \begin{itemize}
  \item Good experience (enjoyable) \\ \begin{CJK}{UTF8}{min}良い経験(楽しい)\end{CJK}
      \smallskip
   
   \item Help colleagues \\ \begin{CJK}{UTF8}{min}同僚を手伝う\end{CJK}
      \smallskip
   
   \item Contribute to good cause %(donate time)
    \\ \begin{CJK}{UTF8}{min}良い原因に貢献する\end{CJK}
      \smallskip
   
   \item Learn from other instructors \\ \begin{CJK}{UTF8}{min}他のインストラクターから\\ 学ぶ\end{CJK}
      \smallskip
   
   \item Part of a global community \\ \begin{CJK}{UTF8}{min}世界中でコミュニティー
\end{CJK}
   
  \end{itemize}
    \end{column}
     \end{columns}  
  
  \end{frame}
    \begin{frame}
  
  \frametitle{Benefits for ``Software Carpentry'' instructors \\
  \begin{CJK}{UTF8}{min}「ソフトウェア・カーペントリー」の利点\end{CJK}}  
  
  \vskip-1cm
\begin{columns}[T] % contents are top vertically aligned
  \begin{column}[T]{7.5cm} %
       \end{column} \begin{column}[T]{4cm} %
      
\begin{center}
      \resizebox{0.5 \columnwidth}{!}{
	\includegraphics{software-carpentry.png}
      }   
  \end{center}
       \end{column}
     \end{columns}  
  
``Selfish reasons''    \\ \begin{CJK}{UTF8}{min}利己的な理由\end{CJK}
  
  \begin{itemize}
  \item Find jobs in future \\ \begin{CJK}{UTF8}{min}将来の仕事探し\end{CJK}
      \smallskip
   
   \item Less questions to answer from colleagues \\ \begin{CJK}{UTF8}{min}同僚からの質問が減る\end{CJK}
      \smallskip
   
   \item Events are fun (and there's free food) \\ \begin{CJK}{UTF8}{min}イベントは楽しい(無料食べ物)\end{CJK}
      \smallskip
   
   \item Practice for public speaking or teaching \\ \begin{CJK}{UTF8}{min}人の前で話す経験や、教えることの練習\end{CJK}
   
  \end{itemize}
  
  \end{frame}
      \begin{frame}
  
  \frametitle{Benefits for ``Software Carpentry'' instructors \\
  \begin{CJK}{UTF8}{min}「ソフトウェア・カーペントリー」の利点\end{CJK}}  
  
  \vskip-1cm
\begin{columns}[T] % contents are top vertically aligned
  \begin{column}[T]{7.5cm} %
       \end{column} \begin{column}[T]{4cm} %
      
\begin{center}
      \resizebox{0.5 \columnwidth}{!}{
	\includegraphics{software-carpentry.png}
      }   
  \end{center}
       \end{column}
     \end{columns}  
  
Because of Software Carpentry I am:   \\ \begin{CJK}{UTF8}{min}ソフトウェア・カーペントリーから、私は:\end{CJK}
  
  \begin{itemize}
  \item A better programmer \\ \begin{CJK}{UTF8}{min}プログラミングがより上手になる\end{CJK}
      \smallskip
   
   \item A better teacher\\ \begin{CJK}{UTF8}{min}教えることがより上手になる\end{CJK}
      \smallskip
   
   \item A better public speaker \\ \begin{CJK}{UTF8}{min}人の前で話すことがより上手になる\end{CJK}
      \smallskip
   
   \item Better at teamwork \\ \begin{CJK}{UTF8}{min}チームワークがより上手になる\end{CJK}
      \smallskip
   
   \item Better at organising events \\ \begin{CJK}{UTF8}{min}イベントの開催がより上手になる\end{CJK}
   
  \end{itemize}
  
  \end{frame}
  
  \begin{frame}
   
   \begin{center}
   
   \LARGE{Can we bring it to Japan?}\\
   \LARGE{\begin{CJK}{UTF8}{min}「ソフトウェア・カーペントリー」は日本で出来る?\end{CJK}}
   
   
   \begin{center}
      \resizebox{0.5 \columnwidth}{!}{
	\includegraphics{software-carpentry.png}
      }   
  \end{center}
   \end{center}

  \end{frame}
   \begin{frame}

\begin{center}
      \resizebox{0.5 \columnwidth}{!}{
	\includegraphics{we_want_you.jpg}
      }   
      
      \Large{We want you} \\ \begin{CJK}{UTF8}{min}\Large{我々は君が欲しい}\end{CJK}
  \end{center}

\end{frame}
  \begin{frame}
   
   \begin{center}
   
   \LARGE{How do I get involved?}\\
   \LARGE{\begin{CJK}{UTF8}{min}「ソフトウェア・カーペントリー」にどのように参加しますか?\end{CJK}}
   
   \begin{center}
      \resizebox{0.5 \columnwidth}{!}{
	\includegraphics{software-carpentry.png}
      }   
  \end{center}
   \end{center}

  \end{frame}
  \begin{frame}

 \begin{center}
 \resizebox{0.25 \textwidth}{!}{
	\includegraphics{software-carpentry.png}
      } 
\\    \texttt{\#}swcarpentry \texttt{@}swcarpentry   
\bigskip
      
 \resizebox{0.25 \textwidth}{!}{
	\includegraphics{data-carpentry.png}
      }  
      \\ \texttt{@}datacarpentry   
\bigskip
       
       
       \resizebox{0.33 \textwidth}{!}{
	\includegraphics{TheCarpentries-opengraph.png}
      }    
      
       \\ \texttt{@}thecarpentries   
 \end{center}
  
\end{frame}    
\fi
  \begin{frame}
  
  \frametitle{Please join ``Software Carpentry''  \\
  \begin{CJK}{UTF8}{min}「ソフトウェア・カーペントリー」参加してください\end{CJK}}  
  
  \vskip-0.5cm
\begin{columns}[T] % contents are top vertically aligned
  \begin{column}[T]{7.5cm} %
       \end{column} \begin{column}[T]{4cm} %
      
\begin{center}
      \resizebox{0.5 \columnwidth}{!}{
	\includegraphics{software-carpentry.png}
      }   
  \end{center}
       \end{column}
     \end{columns}  
  \vskip-1cm  
  
  \begin{itemize}
  \item To join the organisation \\ \begin{CJK}{UTF8}{min}組織に参加するには\end{CJK}
  
  \url{https://carpentries.org/join/}
      \smallskip
      
   \begin{itemize}
 \item GitHub
 
 \url{https://github.com/swcarpentry}
      \smallskip
   
   \item Mailing list (Topicbox) \begin{CJK}{UTF8}{min}メール\end{CJK}
   
   \url{https://carpentries.topicbox.com}
  %\smallskip
   
   %\item Newsletter \begin{CJK}{UTF8}{min}ニュースレター\end{CJK}
   \end{itemize}

   \item To become an instructor (online or in-person training) \\ \begin{CJK}{UTF8}{min}インストラクターになるには(ネットでか人で)\end{CJK}
   
   \url{https://carpentries.org/become-instructor/}
      \smallskip
   
   \item To help with the Japanese translations (contact me) \\ \begin{CJK}{UTF8}{min}日本語の翻訳を手伝うには(私に連絡)\end{CJK}
   
   \url{https://github.com/TomKellyGenetics} \url{tom.kelly@riken.jp} \texttt{@}tomkXY
   
  \end{itemize}
  
  \end{frame}

  \begin{frame}
   
   \begin{center}
   
   \LARGE{What would we need to do this?}\\
   \LARGE{\begin{CJK}{UTF8}{min}「ソフトウェア・カーペントリー」をするには何が必要ですか?\end{CJK}}
   
   
   \begin{center}
      \resizebox{0.5 \columnwidth}{!}{
	\includegraphics{software-carpentry.png}
      }   
  \end{center}
   \end{center}

  \end{frame}
  
    \begin{frame}
  
  \frametitle{Please join ``Software Carpentry''  \\
  \begin{CJK}{UTF8}{min}「ソフトウェア・カーペントリー」参加してください\end{CJK}}  
  
  \vskip-0.5cm
\begin{columns}[T] % contents are top vertically aligned
  \begin{column}[T]{7.5cm} %
       \end{column} \begin{column}[T]{4cm} %
      
\begin{center}
      \resizebox{0.5 \columnwidth}{!}{
	\includegraphics{software-carpentry.png}
      }   
  \end{center}
       \end{column}
     \end{columns}  
  \vskip-1cm  
  
  \begin{itemize}
  \item Japanese translation of the lessons \\ \begin{CJK}{UTF8}{min}レッスンを日本語に翻訳する\end{CJK}
  
  demo: \url{https://tomkellygenetics.github.io/git-novice/ja/index/index.html}
      \smallskip
      
   \begin{itemize}
 \item Team to translate core lessons \\ \begin{CJK}{UTF8}{min}コアレッスンを翻訳するチーム
\end{CJK}
      \smallskip
   
   \item Maintainers to keep it up to date \\ \begin{CJK}{UTF8}{min}メンテナは最新の状態に保つ\end{CJK}
   \end{itemize}

   \item Instructors in Japan (who speak Japanese) \\ \begin{CJK}{UTF8}{min}日本でインストラクター(英語や日本語を話せる) \end{CJK}
      \smallskip
   
   \item Support from institutions \\ \begin{CJK}{UTF8}{min}学会からのサポート \end{CJK}
      \smallskip
   
   \item Then we can plan a workshop or conference \\ \begin{CJK}{UTF8}{min}そして、ワークショップや会議を計画できる \end{CJK}
   
   \end{itemize}
  
  \end{frame}
  \begin{frame}
   
\vskip1cm   
   
   \begin{center}
   
   \large{Please consider to volunteer}\\
   \Large{\begin{CJK}{UTF8}{min}やってみたいかたはいませんか?\end{CJK}}
   
   Contact: \url{tom.kelly@riken.jp} Twitter: \texttt{@}tomkXY GitHub: TomKellyGenetics
   
   \bigskip   
   
   \large{Let's do our best for the community}\\
   \Large{\begin{CJK}{UTF8}{min}コミュニティーのために頑張りましょう!\end{CJK}}
   
      \large{Thank you for your attention}\\
   \Large{\begin{CJK}{UTF8}{min}よろしくお願いいたします\end{CJK}}
   
   
   \begin{center}
      \resizebox{0.33 \columnwidth}{!}{
	\includegraphics{software-carpentry.png}
      }   
  \end{center}
   \end{center}
   
     \vskip-4cm
\begin{columns}[T] % contents are top vertically aligned
  \begin{column}[T]{7.5cm} %
       \end{column} \begin{column}[T]{4cm} %
      
\begin{center}
      \resizebox{1 \columnwidth}{!}{
	\includegraphics{Doraemon.pdf}
      }   
  \end{center}
       \end{column}
     \end{columns}  

  \end{frame}
    
 \end{document}